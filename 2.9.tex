\documentclass{report}

\usepackage{amsmath}
\usepackage{amssymb}
\usepackage{mathtools}
\usepackage[margin=1in]{geometry}

\DeclarePairedDelimiter\abs{\lvert}{\rvert}%

 \newcommand{\?}{\stackrel{?}{=}}
  \newcommand{\<}{\stackrel{?}{<}}

\begin{document}
\section*{2.9}

\begin{align}
i_1 &= (a, b)\\
i_2 &= (c, d)\\
\\
w_{i_1} &= \frac{1}{2}(b-a)\\
w_{i_2} &= \frac{1}{2}(d-c)\\
\end{align}\\*
To show that the width of the sum of two intervals is a function of widths of the two intervals:
\begin{align}
i_1+i_2 &= (a+c, b+d)\\
w_{i_1+i_2} &= \frac{1}{2}(b+d - (a+c))\\
\\
w_{i_1} + w_{i_2} &= \frac{1}{2}(b-a) + \frac{1}{2}(d-c)\\
w_{i_1} + w_{i_2} &= \frac{1}{2}(b-a + d-c)\\
w_{i_1} + w_{i_2} &= \frac{1}{2}(b + d - a -c)\\
w_{i_1} + w_{i_2} &= \frac{1}{2}(b+ d -(a+c))\\
\therefore w_{i_1} + w_{i_2} &= w_{i_1+i_2} \:\:\; \checkmark\\
\end{align}\\
Similarly, for the difference of two intervals:
\begin{align}
i_1-i_2 &= (a-d, b-c)\\
w_{i_1-i_2} &= \frac{1}{2}(b-c - (a-d))\\
w_{i_1-i_2} &= \frac{1}{2}(b-c - a+d))\\
w_{i_1-i_2} &= \frac{1}{2}(b+d - (a+c))\\
\therefore w_{i_1} + w_{i_2} &= w_{i_1-i_2} \:\:\; \checkmark\\
\end{align}\\*
\pagebreak

To show that the width of the product of two intervals is not a function of the widths of the two intervals alone:
\begin{align}
i_1 &= (0.25, 0.75)\\
i_2 &= (1, 2)\\
w_{i_1} &= \frac{1}{2}(0.75-0.25) = 0.25\\
w_{i_2} &= \frac{1}{2}(2-1)=0.5\\
\\
i_3 &= (1, 1.5)\\
i_4 &= (1, 2)\\
w_{i_3} &= \frac{1}{2}(1.5-1) = 0.25\\
w_{i_4} &= \frac{1}{2}(2-1)=0.5\\
\end{align}

Here we have two pairs of intervals with the same widths. However:
\begin{align}
i_1\times i_2 &= (min(p_1, p_2, p_3, p_4), max(p_1, p_2, p_3, p_4))\\
where:\\
p_1 &= (0.25)(1) = 0.25\\
p_2 &= (0.25)(2) = 0.5\\
p_3 &= (0.75)(1) = 0.75\\
p_4 &= (0.75)(2) = 1.5\\
\\
i_1\times i_2 &= (0.25, 1.5)\\
w_{i_1\times i_2} &= \frac{1}{2}(1.5 - 0.25)\\
w_{i_1\times i_2} &= 0.625\\
\\
i_3\times i_4 &= (min(p_1, p_2, p_3, p_4), max(p_1, p_2, p_3, p_4))\\
where:\\
p_1 &= (1)(1) = 1\\
p_2 &= (1)(2) = 2\\
p_3 &= (1.5)(1) = 1.5\\
p_4 &= (1.5)(2) = 3\\
\\
i_1\times i_2 &= (1, 3)\\
w_{i_3\times i_4} &= \frac{1}{2}(3 - 1)\\
w_{i_3\times i_4} &= 1\\
\\
w_{i_1\times i_2} &\neq w_{i_3\times i_4}\\
\end{align}\\
$\therefore$ the width of the product of two intervals cannot be a function of only the widths of the two intervals.
\pagebreak

We can use the same two pairs of intervals to show that the width of the quotient of two intervals is not a function of the widths of the two intervals alone:
\begin{align}
\frac{i_1}{i_2} &= i_1\times i_2'\\
where:\\
i_2' &=  \left(\frac{1}{(\operatorname{upper-bound} i_2)}, \frac{1}{(\operatorname{lower-bound} i_2)}\right)\\
i_2' &=  \left(\frac{1}{2}, \frac{1}{1}\right) = (0.5, 1)\\
\\
i_1\times i_2' &= (min(p_1, p_2, p_3, p_4), max(p_1, p_2, p_3, p_4))\\
where:\\
p_1 &= (0.25)(0.5) = 0.125\\
p_2 &= (0.25)(1) = 0.25\\
p_3 &= (0.75)(0.5) = 0.375\\
p_4 &= (0.75)(1) = 0.75\\
\\
i_1\times i_2' &= (0.125, 0.75)= \frac{i_1}{i_2}\\
w_{i_1/ i_2} &= \frac{1}{2}(0.75 - 0.125) = 0.3125\\
\\
\frac{i_3}{i_4} &= i_1\times i_4'\\
where:\\
i_4' &=  \left(\frac{1}{(\operatorname{upper-bound} i_2)}, \frac{1}{(\operatorname{lower-bound} i_2)}\right)\\
i_4' &=  \left(\frac{1}{2}, \frac{1}{1}\right) = (0.5, 1)\\
\\
i_3\times i_4' &= (min(p_1, p_2, p_3, p_4), max(p_1, p_2, p_3, p_4))\\
where:\\
p_1 &= (1)(0.5) = 0.5\\
p_2 &= (1)(1) = 1\\
p_3 &= (1.5)(0.5) = 0.75\\
p_4 &= (1.5)(1) = 1.5\\
\\
i_3\times i_4' &= (0.5, 1.5) = \frac{i_3}{i_4}\\
w_{i_3/ i_4} &= \frac{1}{2}(1.5 - 0.5) = 0.5\\
\\
w_{i_1/i_2} &\neq w_{i_3/i_4}\\
\end{align}\\
$\therefore$ the width of the quotient of two intervals cannot be a function of only the widths of the two intervals.

\end{document}